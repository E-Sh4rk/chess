\documentclass[11pt]{article}
\usepackage[francais]{babel}
\usepackage[utf8]{inputenc}
\usepackage{mathtools,amssymb,amsthm,thmtools}
\usepackage{hyperref}
\usepackage{listings}
\usepackage{amssymb}
\usepackage[left=2cm,right=2cm,top=2cm,bottom=2cm]{geometry}
\title{Projet Programmation 2}
\author{Mickaël LAURENT, Maher MALLEM}
\date{Mai 2017}
%\author{}
%\date{}
\begin{document}
\renewcommand{\labelitemi}{-}
\maketitle

\-

\section{Fonctionnalités}

Voici les fonctionnalités ajoutées :\newline
\begin{itemize}
\item Le jeu supporte maintenant un nombre quelconque de périodes pour la pendule (fonctionnalité pas encore implémentée la dernière fois).
\item Quelques améliorations graphiques (texte dans le canvas pour informer l'utilisateur quand l'IA est en train de jouer, etc).
\item Ajout de paramètres d'IA dans les options.
\item Ajout d'une IA implémentant Alpha-Beta. Elle peut fonctionner sur les différentes variantes de jeu. Elle dispose d'une fonction d'évaluation prenant en compte les pièces restantes ainsi que leur position : pour chaque type de pièce, elle possède un tableau qui associe une valeur à chaque case de l'échiquier.
\item Ajout d'une IA utilisant le moteur de GNU-Chess. Les limites de temps sont gérées, ainsi que la reprise de partie à n'importe quel moment. On peut faire s'affronter GNU-Chess contre lui-même, l'opposer à une autre IA ou alors jouer contre lui. Les variantes Janus et Capablanca ne sont cependant pas gérées par GNU-Chess.
\end{itemize}

\section{La classe Rules}
Afin d'implémenter une IA plus avancé, il a été nécessaire d'ajouter une classe intermédiaire entre la classe \textit{Board} (qui s'occupe uniquement d'abstraire la structure de donnée utilisée pour représenter l'échiquier et de fournir des accès à ce dernier) et \textit{Game} (qui représente le jeu dans son ensemble, avec les règles, les joueurs, l'horloge). En effet, une classe \textit{Rules} a été ajoutée : elle représente l'échiquier muni des règles du jeu (déplacement, conditions de fin, mais pas de timer ni de joueur). C'était la classe \textit{Game} qui implémentait ces règles auparavant, mais désormais la classe \textit{Game} ne fait qu'hériter de ces règles. 

\-

Cette petite restructuration a été nécessaire afin de pouvoir simuler plusieurs coups en avance sans réellement modifier le jeu.
En effet, il est nécessaire pour réaliser ces simulations de copier l'instance actuelle du jeu et de réaliser les mouvements à simuler sur cette copie, afin de ne pas modifier le jeu réel. Cependant, copier la classe \textit{Game} n'est pas une bonne idée :
elle contient également les joueurs et le timer nécessaire pour l'horloge, et nous ne souhaitons pas copier ni modifier ces éléments !
Grâce à la séparation de \textit{Game} et \textit{Rules}, on peut désormais copier uniquement une instance de \textit{Rules}, et donc ne pas s'encombrer des joueurs et de l'horloge.

\section{L'IA Alpha-Beta}

L'IA Alpha-Beta développée est une sous-classe de SynchPlayer, donc un joueur comme un autre qui renvoie un coup lorsqu'on lui demande de jouer. Elle repose, comme on s'y attend, sur la combinaison de deux éléments distincts : un algorithme de parcours des configurations, et une fonction d'évaluation des positions de jeu.

\-

L'algorithme de parcours est directement implémenté comme une méthode de la classe. A ce jour, le parcours suit l'algorithme Minimax avec l'élagage Alpha-Bêta implémenté de manière fidèle - sans autre optimisation. Pour chaque coup possible, on copie l'instance courante de \textit{Rules}, on simule le coup dans la copie et on appelle récursivement la procédure dessus jusqu'à la profondeur indiquée, à laquelle on évalue alors la configuration.

\-

La fonction d'évaluation est donnée en paramètre de classe. De cette manière, on a une liberté importante sur la fonction d'évaluation choisie. Dans l'implémentation, une fonction d'évaluation quelconque est caractérisée par la classe abstraite EvalFunc, ayant une méthode qui prend une instance de Rules en entrée et renvoie un entier : la valeur de la configuration du jeu. Actuellement, l'accent a été mis sur une fonction d'évaluation en particulier : EvalStd.

\-

EvalStd reprend les grandes lignes de la fonction d'évaluation \textit{simple} donnée en lien dans le sujet : la répartition des poids vient d'observations locales, par exemple éviter d'avoir des cavaliers le long des bords de l'échiquier et les préferer au centre. Notons que EvalStd distingue deux instants dans une partie : le milieu et la fin de partie. En effet, le roi se doit d'avoir une mobilité différente selon l'avancement de la partie. Cette distinction est opérée par la méthode \textit{isEndGame} : on est en fin de partie si et seulement si chaque joueur a perdu sa dame ou n'a qu'un nombre restreint d'autres pièces encore en jeu.

\-

L'IA comporte trois paramètres, tous modifiables depuis le menu des options de l'interface graphique :
\begin{itemize}
\item \textit{maxDepth} : la profondeur de l'algorithme de recherche, qui peut se voir aussi comme le niveau de difficulté.
\item \textit{attCoef} : plus ce coefficient est élevé, plus l'IA cherchera à mettre en echec son adversaire par tous les moyens. On l'observe particulièrement bien avec une profondeur de 0.
\item \textit{defCoef} : plus ce coefficient est élevé, plus l'IA cherchera à éviter d'être mise en échec.
\end{itemize}

\-

Au niveau des performances réelles de l'IA :
\begin{itemize}
\item Profondeur 0 ou 1 : L'IA joue quasi-instantanement.
\item Profondeur 2 : L'IA est rapide (en général pas plus de 5 secondes par coup). Son jeu n'est bien sûr pas parfait, mais on peut déjà observer la prise de pièces adverses non protégées, ainsi que la prise en compte de la promotion et du roque.
\item Profondeur 3 : L'attente est plus longue, mais convenable par rapport au temps de jeu moyen aux échecs : en général moins de 50 secondes par coup, et jusqu'à 2 minutes pour le pire cas observé. Le jeu de l'IA est plus fort qu'en profondeur 2 : on observe notamment des protections de pièces.
\item Profondeur $\geq$ 4 : L'IA est trop longue pour être jouable.
\end{itemize}

\section{La classe CECP\_Player}

Cette classe implémente un \textit{Player} communiquant avec le protocole CECP. Seul GNU-Chess est supporté pour le moment.
Un processus de \textsl{gnuchess} (avec l'option -x) est démarré lors de l'initialisation du joueur (méthode \textit{init}),
puis \textit{CECP\_Player} maintient une communication avec ce processus sur un nouveau thread à travers trois pipes (input, output, error).

\-

Pour initialiser le jeu, la commande \textit{new} est envoyée. Si le jeu est déjà entamé (i.e. si on le jeu n'est pas dans la configuration initiale), un fichier pgn contenant l'historique des mouvements déjà réalisés est écrit dans un répertoire temporaire,
sous un format un peu différent de la notation algébrique abrégée utilisée lors des sauvegardes car \textit{gnuchess} utilise une variante de ce format. Il est ensuite chargé dans \textit{gnuchess} via la commande \textit{pgnload}, puis le fichier pgn est supprimé.

\-

Lorsque c'est à \textit{CECP\_Player} de jouer, si un autre joueur vient de jouer son tour alors sont son coup est communiqué à \textit{gnuchess}, et ce dernier va jouer automatiquement son tour. Dans le cas inverse, on force \textit{gnuchess} à jouer grâce à la commande \textit{go}. On parse ensuite le résultat et on le communique à la classe \textit{Game}.

\-

\textit{GNU-Chess} est également informé du temps qu'il lui reste pour jouer grâce à la commande \textit{time}. Le temps que lui indique \textit{CECP\_Player} est le minimum entre le temps réel qu'il reste (si l'horloge est activée pour le jeu en cours) et le temps limite de réflexion accordé à \textit{gnuchess}, modifiable dans les options.

\-

Lorsque la partie est terminé ou qu'elle soit se suspendre (méthode \textit{stop}), le processus \textit{gnuchess} est tué et le thread
s'arrète.

\-

\end{document}
